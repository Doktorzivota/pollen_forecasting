\section{Rezultati}


\begin{frame}
    \frametitle{Rezultati imputacije podataka}

    \begin{columns}[T, totalwidth=\textwidth]

        % Leva kolona: slika
        \begin{column}{0.5\textwidth}
            \begin{figure}
                \centering
                \includegraphics[width=\textwidth]{images/AMBROZIJA_PANČEVO_imputed_concentration.png}
                \caption{Primer imputacije vremenske serije}
            \end{figure}
        \end{column}
        
        % Desna kolona: tabela
        \begin{column}{0.48\textwidth}
            \centering
            \small
            \vspace{0.5cm}
            \begin{tabular}{lcc}
                \toprule
                \textbf{Metod} & \textbf{Bez meteo} & \textbf{Sa meteo} \\
                \midrule
                Naivna interpolacija & 0.309 & – \\
                IDW interpolacija & 1.085 & – \\
                Kriging + Box–Cox & 0.260 & 0.262 \\
                \bottomrule
            \end{tabular}
            \captionof{table}{Uporedni rezultati imputacije ambrozije (RMSLE).}
        \end{column}

    \end{columns}
\end{frame}


\begin{frame}
    \frametitle{Predikcija 7 dana unapred — SARIMAX}

    \begin{figure}[t]
        \centering
        \subfloat{
            \includegraphics[width=0.48\textwidth]{images/AMBROZIJA_POŽAREVAC_7d_nometeo_sarimax.png}
        }
        \hfill
        \subfloat{
            \includegraphics[width=0.48\textwidth]{images/AMBROZIJA_POŽAREVAC_7d_meteo_sarimax.png}
        }
        \caption{Primer predikcije 7 dana unapred (SARIMAX model sa logaritamskom transformacijom).}
    \end{figure}

    \vspace{0.1cm}

    \centering
    \small
    \begin{tabular}{lcc}
        \toprule
        \textbf{Model} & \textbf{Bez meteo} & \textbf{Sa meteo} \\
        \midrule
        SARIMAX & 2.7 & 2.6 \\
        Prophet & 3.6 & 3.3 \\
        Random Forest & 4.0 & 4.0 \\
        \bottomrule
    \end{tabular}
    \captionof{table}{Prosečna greška početka sezone (u danima) za ambroziju.}

\end{frame}



\begin{frame}
    \frametitle{Predikcija 1 dan unapred — Random Forest}

    \begin{figure}[t]
        \centering
        \subfloat{
            \includegraphics[width=0.48\textwidth]{images/AMBROZIJA_POŽAREVAC_1d_nometeo_rf.png}
        }
        \hfill
        \subfloat{
            \includegraphics[width=0.48\textwidth]{images/AMBROZIJA_POŽAREVAC_1d_meteo_rf.png}
        }
        \caption{Primer predikcije 1 dan unapred (Random Forest model sa logaritamskom transformacijom).}
    \end{figure}

    \vspace{0.1cm}
    \centering
    \small
    \begin{tabular}{lcc}
        \toprule
        \textbf{Model} & \textbf{Bez meteo} & \textbf{Sa meteo} \\
        \midrule
        SARIMAX  & 0.912 & 0.918 \\
        Prophet & 0.887 & 0.906 \\
        Random Forest & 0.911 & 0.922 \\
        \bottomrule
    \end{tabular}
    \captionof{table}{Prosečni F1-skor modela za predikciju visokih koncentracija polena ambrozije.}
\end{frame}

\begin{frame}
    \frametitle{Značaj karakteristika u Random Forest modelu}

    \begin{columns}[T, totalwidth=\textwidth]
        \begin{column}{0.55\textwidth}
            \textbf{Analiza značaja ulaznih karakteristika:}

            \begin{itemize}
                \item Najuticajnije promenljive: 
                \begin{itemize}
                    \item \textbf{Vremensko kašnjenje koncentracije polena do 3 dana}
                    \item \textbf{Prosečna koncentracija u prethodnih 7 dana}
                    \item \textbf{Temperatura i relativna vlažnost vazduha}
                    \item \textbf{Furijeovi redovi}
                \end{itemize}
                \item Pokazuje koje meteorološke i istorijske komponente najviše utiču na predikciju.
            \end{itemize}
        \end{column}

        \begin{column}{0.45\textwidth}
            \centering
            \includegraphics[width=\textwidth]{images/feature_importance_AMBROZIJA_POŽAREVAC.png}
            \captionof{figure}{Značaj karakteristika u Random Forest modelu.}
        \end{column}
    \end{columns}

\end{frame}



\begin{frame}
    \frametitle{Uporedna evaluacija modela — RMSLE}

    \centering
    \small
    \begin{tabular}{lcccc}
        \toprule
        & \multicolumn{2}{c}{\textbf{1 dan unapred}} & \multicolumn{2}{c}{\textbf{7 dana unapred}} \\
        \cmidrule(lr){2-3} \cmidrule(lr){4-5}
        \textbf{Model} & \textbf{Bez meteo} & \textbf{Sa meteo} & \textbf{Bez meteo} & \textbf{Sa meteo} \\
        \midrule
        SARIMAX & 0.45 & 0.42 & 0.50 & 0.46 \\
        Prophet & 0.50 & 0.46 & 0.59 & 0.59 \\
        Random Forest & 0.46 & 0.41 & 0.58 & 0.55 \\
        Naivna predikcija & 0.54 & -- & 0.78 & -- \\
        \bottomrule
    \end{tabular}

    \vspace{0.3cm}
    \captionof{table}{RMSLE rezultati modela — ambrozija, Požarevac.}
\end{frame}
